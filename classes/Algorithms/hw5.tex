\documentclass[11pt,letterpaper]{article}
\usepackage{amssymb,amsmath,graphicx,url}
\pdfpagewidth=8.5truein
\pdfpageheight=11.0truein
\setlength{\topmargin}{-.5in}

\setlength{\textheight}{8.875in}
\setlength{\textwidth}{6.5in}
\setlength{\oddsidemargin}{-.1875in}  % Centers text.
\setlength{\evensidemargin}{-.1875in}
\usepackage{graphicx}
\begin{document}
\subsection*{Homework 5: CSCI 6212 Algorithms, due: Beginning of class, Nov. 22, 2019}
    \paragraph{Collaboration Policies:} This is not a group homework, you are required to do this homework yourself.  However, you are {\em allowed} to use or search any passive online resource for information about how to solve the problem, and discuss the problems with classmates.  
    You are {\em not allowed} to use active web resources (like Stack Overflow) where you post questions and ask for responses, post these questions to a "work for hire" site where someone else does them for you, or take *any* written notes from discussions with classmates or others. 

\paragraph{Problems:}
\begin{enumerate}

\item In class we talked about the greedy algorithm for the 
vertex cover problem:a
\begin{verbatim}
(1) picks the node that touches the most edges,
(2) puts that node into the Vertex Cover, 
(3) deletes all those edges from the graph, and
(4) if there are any edges left in the graph, GOTO 1.
\end{verbatim}

In class we claimed that it can have an arbitrarily bad approximation ratio.  In this problem your job is to construct examples where the greedy algorithm does badly.  
To help you, any time where the algorithm gets to choose (for example, if there are two nodes that each touch 4 edges), you can pick for the algorithm (forcing it to choose the "wrong" choice).
\begin{enumerate}
\item Show a graph with 5 nodes where the greedy algorithm chooses 3 nodes but only 2 are necessary.  Explain (in a sentence or two) why the algorithm gets the wrong answer.
\item Show a specific graph with the largest ratio of greedy-solution vs. optimal solution that you can create.  Explain your construction; the reasoning behind how you constructed the graph and which nodes are chosen in what order.  (Hint, in a bi-partite graph, the vertex cover can always consist of all the nodes on one side of the graph)
\end{enumerate}
\item Let NeighborSet be the problem defined as follows. Given a graph $G(V,E)$ and a
number $k$, is there a way to select a set $NS \subseteq V$ with $|NS| = k$ such that every vertex in
the graph is either in NS or is connect by an edge to a vertex in NS. Show that NeighborSet is NP-complete.
\end{enumerate}
\end{document}
